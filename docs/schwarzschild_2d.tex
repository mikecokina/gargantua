\documentclass[11pt,a4paper]{article}

\usepackage{amsmath,amssymb}
\usepackage{physics}
\usepackage{geometry}
\usepackage{hyperref}
\usepackage{bm}

\geometry{margin=2.5cm}

\title{Gargantua\\
\large Physical and Mathematical Foundations of Photon Ray Tracing\\
in Schwarzschild Spacetime}
\author{}
\date{}

\begin{document}
\maketitle

\tableofcontents
\newpage

% ============================================================
\section{Introduction}

This document provides a complete physical and mathematical derivation
of the equations implemented in the \texttt{Gargantua} ray tracer.
The goal is not only to present final formulas, but to show
\emph{why each equation is introduced and how it follows from General Relativity}.

The document is written as a self-contained set of notes,
intended to bridge the gap between theoretical GR and practical numerical implementation.

We focus on:
\begin{itemize}
  \item null geodesics in Schwarzschild spacetime,
  \item reduction of spacetime geodesics to an orbit equation,
  \item geometric construction of initial conditions from ray data,
  \item numerical integration using Runge--Kutta methods.
\end{itemize}

Geometric units $G=c=1$ are used throughout.

% ============================================================
\section{Spacetime Interval and Metric}

In General Relativity, gravity is not a force but a manifestation of spacetime geometry.
This geometry is encoded in the metric tensor $g_{\mu\nu}$.

The invariant spacetime interval between two infinitesimally close events is

\begin{equation}
ds^2 = g_{\mu\nu}\,dx^\mu\,dx^\nu.
\end{equation}

This quantity plays the role analogous to distance in Euclidean space.
All physical predictions must be invariant under coordinate transformations,
which is why the metric formulation is fundamental.

A particle trajectory is a curve $x^\mu(\lambda)$ parametrized
by an affine parameter $\lambda$.

% ============================================================
\section{Why Extremize the Action}

Free particles move along paths that extremize the spacetime interval.
This principle generalizes the principle of least action from classical mechanics.

The action is defined as

\begin{equation}
S = \int ds.
\end{equation}

The actual physical trajectory is the one for which $\delta S = 0$.

For computational purposes, it is inconvenient to work with the square root in $ds$.
Fortunately, extremizing $S=\int ds$ is equivalent to extremizing

\begin{equation}
S = \int \mathcal{L}\,d\lambda,
\qquad
\mathcal{L} = \frac{1}{2} g_{\mu\nu}
\frac{dx^\mu}{d\lambda}
\frac{dx^\nu}{d\lambda}.
\end{equation}

This quadratic Lagrangian produces the same geodesic equations
and greatly simplifies the mathematics.

The parameter $\lambda$ is called an \emph{affine parameter}.
For photons, it does not correspond to proper time,
but still parametrizes the curve consistently.

% ============================================================
\section{Schwarzschild Spacetime}

The Schwarzschild metric describes spacetime outside a static,
spherically symmetric mass $M$.

In Schwarzschild coordinates $(t,r,\theta,\phi)$:

\begin{equation}
ds^2 =
-\left(1-\frac{2M}{r}\right)dt^2
+\left(1-\frac{2M}{r}\right)^{-1}dr^2
+r^2 d\theta^2
+r^2\sin^2\theta\,d\phi^2.
\end{equation}

Due to spherical symmetry, any geodesic can be rotated
into the equatorial plane:

\begin{equation}
\theta = \frac{\pi}{2}.
\end{equation}

The metric then reduces to

\begin{equation}
ds^2 =
-\left(1-\frac{2M}{r}\right)dt^2
+\left(1-\frac{2M}{r}\right)^{-1}dr^2
+r^2 d\phi^2.
\end{equation}

The event horizon is located at

\begin{equation}
r_s = 2M.
\end{equation}

% ============================================================
\section{Geodesic Lagrangian}

Substituting the Schwarzschild metric into the quadratic action yields

\begin{equation}
\mathcal{L}
=
\frac{1}{2}\left[
-\left(1-\frac{2M}{r}\right)\dot{t}^2
+\left(1-\frac{2M}{r}\right)^{-1}\dot{r}^2
+r^2\dot{\phi}^2
\right],
\end{equation}

where dots denote derivatives with respect to the affine parameter $\lambda$.

For photons (null geodesics), the constraint

\begin{equation}
ds^2 = 0
\end{equation}

must hold along the trajectory.

% ============================================================
\section{Conserved Quantities from Symmetries}

Because the Lagrangian does not depend explicitly on $t$ or $\phi$,
their conjugate momenta are conserved.
This is a direct consequence of Noether's theorem.

\subsection{Energy}

\begin{equation}
p_t = \frac{\partial\mathcal{L}}{\partial\dot{t}}
= -\left(1-\frac{2M}{r}\right)\dot{t}.
\end{equation}

We define the conserved energy

\begin{equation}
E = \left(1-\frac{2M}{r}\right)\dot{t}.
\end{equation}

\subsection{Angular Momentum}

\begin{equation}
p_\phi = \frac{\partial\mathcal{L}}{\partial\dot{\phi}}
= r^2\dot{\phi}.
\end{equation}

We define the conserved angular momentum

\begin{equation}
L = r^2\dot{\phi}.
\end{equation}

% ============================================================
\section{Radial Equation of Motion}

For null geodesics, impose the constraint

\begin{equation}
g_{\mu\nu}\dot{x}^\mu\dot{x}^\nu = 0.
\end{equation}

Substituting the metric and conserved quantities yields

\begin{equation}
\dot{r}^2
=
E^2
-
\left(1-\frac{2M}{r}\right)\frac{L^2}{r^2}.
\end{equation}

This equation describes radial motion in an effective potential.

% ============================================================
\section{Eliminating the Affine Parameter}

The shape of the trajectory does not depend on how it is parametrized.
We therefore eliminate $\lambda$ in favor of the angular coordinate $\phi$.

Using the chain rule:

\begin{equation}
\frac{dr}{d\phi}
=
\frac{dr/d\lambda}{d\phi/d\lambda}
=
\frac{\dot{r}}{\dot{\phi}}.
\end{equation}

Since $L=r^2\dot{\phi}$:

\begin{equation}
\dot{\phi} = \frac{L}{r^2}.
\end{equation}

Substituting yields

\begin{equation}
\left(\frac{dr}{d\phi}\right)^2
=
\frac{r^4}{L^2}
\left[
E^2
-
\left(1-\frac{2M}{r}\right)\frac{L^2}{r^2}
\right].
\end{equation}

% ============================================================
\section{Substitution $u = 1/r$}

Define the inverse radius

\begin{equation}
u = \frac{1}{r}.
\end{equation}

This substitution simplifies the equation and turns the orbit problem
into a differential equation for $u(\phi)$.

Using

\begin{equation}
\frac{dr}{d\phi} = -\frac{1}{u^2}\frac{du}{d\phi},
\end{equation}

we obtain

\begin{equation}
\left(\frac{du}{d\phi}\right)^2
=
\frac{1}{b^2}
-
u^2
+
2Mu^3,
\qquad
b=\frac{L}{E}.
\end{equation}

Differentiating with respect to $\phi$ gives the second-order orbit equation:

\begin{equation}
\boxed{
\frac{d^2u}{d\phi^2} + u = 3Mu^2.
}
\end{equation}

This equation fully determines photon trajectories.

% ============================================================
\section{Geometric Initial Conditions}

The numerical solver requires initial values

\[
u_0 = u(\phi_0),\qquad
v_0 = \left.\frac{du}{d\phi}\right|_{\phi_0}.
\]

Given initial position $\bm{p}_0$ and direction $\bm{d}$,
we construct a local polar basis around the black hole center $\bm{c}$:

\begin{align}
\hat{\bm{r}} &= \frac{\bm{p}_0-\bm{c}}{|\bm{p}_0-\bm{c}|},\\
\hat{\bm{\phi}} &= (-\hat{r}_y,\hat{r}_x).
\end{align}

Velocity decomposition:

\begin{equation}
\dot{\bm{r}} = \dot{r}\hat{\bm{r}} + r\dot{\phi}\hat{\bm{\phi}}.
\end{equation}

Thus:

\begin{align}
\dot{r} &= \bm{d}\cdot\hat{\bm{r}},\\
r\dot{\phi} &= \bm{d}\cdot\hat{\bm{\phi}}.
\end{align}

Using $u=1/r$, the initial slope is

\begin{equation}
\boxed{
v_0
=
-\frac{1}{r_0}
\frac{\bm{d}\cdot\hat{\bm{r}}}{\bm{d}\cdot\hat{\bm{\phi}}}.
}
\end{equation}

% ============================================================
\section{Numerical Integration: Runge--Kutta 4}

The orbit equation is written as a first-order system:

\begin{align}
\frac{du}{d\phi} &= v,\\
\frac{dv}{d\phi} &= -u + 3Mu^2.
\end{align}

This system is integrated using the classical fourth-order Runge--Kutta method.
RK4 evaluates the slope at multiple points inside each step,
providing high accuracy and stability for smooth trajectories.

% ============================================================
\section{Conclusion}

The Gargantua ray tracer implements photon trajectories
as null geodesics of Schwarzschild spacetime.
Each component of the code corresponds directly
to a well-defined physical and mathematical concept,
making the implementation both accurate and interpretable.

\end{document}
